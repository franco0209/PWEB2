%package list
\documentclass{article}
\usepackage[top=3cm, bottom=3cm, outer=3cm, inner=3cm]{geometry}
\usepackage{multicol}
\usepackage{graphicx}
\usepackage{url}
%\usepackage{cite}
\usepackage{hyperref}
\usepackage{array}
%\usepackage{multicol}
\newcolumntype{x}[1]{>{\centering\arraybackslash\hspace{0pt}}p{#1}}
\usepackage{natbib}
\usepackage{pdfpages}
\usepackage{multirow}
\usepackage[normalem]{ulem}
\useunder{\uline}{\ul}{}
\usepackage{svg}
\usepackage{xcolor}
\usepackage{listings}
\lstdefinestyle{ascii-tree}{
    literate={├}{|}1 {─}{--}1 {└}{+}1 
  }
\lstset{basicstyle=\ttfamily,
  showstringspaces=false,
  commentstyle=\color{red},
  keywordstyle=\color{blue}
}
%\usepackage{booktabs}
\usepackage{caption}
\usepackage{subcaption}
\usepackage{float}
\usepackage{array}

\newcolumntype{M}[1]{>{\centering\arraybackslash}m{#1}}
\newcolumntype{N}{@{}m{0pt}@{}}


%%%%%%%%%%%%%%%%%%%%%%%%%%%%%%%%%%%%%%%%%%%%%%%%%%%%%%%%%%%%%%%%%%%%%%%%%%%%
%%%%%%%%%%%%%%%%%%%%%%%%%%%%%%%%%%%%%%%%%%%%%%%%%%%%%%%%%%%%%%%%%%%%%%%%%%%%
\newcommand{\itemEmail}{fcahua@unsa.edu.pe}
\newcommand{\itemStudent}{Franco Jesus Cahua Soto}
\newcommand{\itemCourse}{Programación Web 2}
\newcommand{\itemCourseCode}{20222171}
\newcommand{\itemSemester}{I}
\newcommand{\itemUniversity}{Universidad Nacional de San Agustín de Arequipa}
\newcommand{\itemFaculty}{Facultad de Ingeniería de Producción y Servicios}
\newcommand{\itemDepartment}{Departamento Académico de Ingeniería de Sistemas e Informática}
\newcommand{\itemSchool}{Escuela Profesional de Ingeniería de Sistemas}
\newcommand{\itemAcademic}{2043 - A}
\newcommand{\itemInput}{}
\newcommand{\itemOutput}{}
\newcommand{\itemPracticeNumber}{3}
\newcommand{\itemTheme}{JavaScript)}
%%%%%%%%%%%%%%%%%%%%%%%%%%%%%%%%%%%%%%%%%%%%%%%%%%%%%%%%%%%%%%%%%%%%%%%%%%%%
%%%%%%%%%%%%%%%%%%%%%%%%%%%%%%%%%%%%%%%%%%%%%%%%%%%%%%%%%%%%%%%%%%%%%%%%%%%%

\usepackage[english,spanish]{babel}
\usepackage[utf8]{inputenc}
\AtBeginDocument{\selectlanguage{spanish}}
\renewcommand{\figurename}{Figura}
\renewcommand{\refname}{Referencias}
\renewcommand{\tablename}{Tabla} %esto no funciona cuando se usa babel
\AtBeginDocument{%
	\renewcommand\tablename{Tabla}
}

\usepackage{fancyhdr}
\pagestyle{fancy}
\fancyhf{}
\setlength{\headheight}{30pt}
\renewcommand{\headrulewidth}{1pt}
\renewcommand{\footrulewidth}{1pt}
\fancyhead[L]{\raisebox{-0.2\height}{\includegraphics[width=3cm]{logo_episunsa.png}}}
\fancyhead[C]{\fontsize{7}{7}\selectfont	\itemUniversity \\ \itemFaculty \\ \itemDepartment \\ \itemSchool \\ \textbf{\itemCourse}}
\fancyhead[R]{\raisebox{-0.2\height}{\includegraphics[width=1.2cm]{abet.png}}}
\fancyfoot[L]{Estudiante Franco Jesus Cahua Soto}
\fancyfoot[C]{\itemCourse}
\fancyfoot[R]{Página \thepage}

% para el codigo fuente
\usepackage{listings}
\usepackage{color, colortbl}
\definecolor{dkgreen}{rgb}{0,0.6,0}
\definecolor{gray}{rgb}{0.5,0.5,0.5}
\definecolor{mauve}{rgb}{0.58,0,0.82}
\definecolor{codebackground}{rgb}{0.95, 0.95, 0.92}
\definecolor{tablebackground}{rgb}{0.8, 0, 0}

\lstset{frame=tb,
	language=bash,
	aboveskip=3mm,
	belowskip=3mm,
	showstringspaces=false,
	columns=flexible,
	basicstyle={\small\ttfamily},
	numbers=none,
	numberstyle=\tiny\color{gray},
	keywordstyle=\color{blue},
	commentstyle=\color{dkgreen},
	stringstyle=\color{mauve},
	breaklines=true,
	breakatwhitespace=true,
	tabsize=3,
	backgroundcolor= \color{codebackground},
}

\begin{document}
	
	\vspace*{10px}
	
	\begin{center}	
		\fontsize{17}{17} \textbf{ Informe de Laboratorio \itemPracticeNumber}
	\end{center}
	\centerline{\textbf{\Large Tema: \itemTheme}}
	%\vspace*{0.5cm}	

	\begin{flushright}
		\begin{tabular}{|M{2.5cm}|N|}
			\hline 
			\rowcolor{tablebackground}
			\color{white} \textbf{Nota}  \\
			\hline 
			     \\[30pt]
			\hline 			
		\end{tabular}
	\end{flushright}	

	\begin{table}[H]
		\begin{tabular}{|x{4.7cm}|x{4.8cm}|x{4.8cm}|}
			\hline 
			\rowcolor{tablebackground}
			\color{white} \textbf{Estudiante} & \color{white}\textbf{Escuela}  & \color{white}\textbf{Asignatura}   \\
			\hline 
			{\itemStudent \par \itemEmail} & \itemSchool & {\itemCourse \par Semestre: \itemSemester \par Código: \itemCourseCode}     \\
			\hline 			
		\end{tabular}
	\end{table}		
	
	\begin{table}[H]
		\begin{tabular}{|x{4.7cm}|x{4.8cm}|x{4.8cm}|}
			\hline 
			\rowcolor{tablebackground}
			\color{white}\textbf{Laboratorio} & \color{white}\textbf{Tema}  & \color{white}\textbf{Duración}   \\
			\hline 
			\itemPracticeNumber & \itemTheme & 04 horas   \\
			\hline 
		\end{tabular}
	\end{table}
	
	\begin{table}[H]
		\begin{tabular}{|x{4.7cm}|x{4.8cm}|x{4.8cm}|}
			\hline 
			\rowcolor{tablebackground}
			\color{white}\textbf{Semestre académico} & \color{white}\textbf{Fecha de inicio}  & \color{white}\textbf{Fecha de entrega}   \\
			\hline 
			\itemAcademic & \itemInput &  \itemOutput  \\
			\hline 
		\end{tabular}
	\end{table}

    \section{TAREA}
	\begin{itemize}	

    \section{URL DE REPOSITORIO GITHUB}
	\begin{itemize}
		\item URL para el Repositorio GitHub.
		\item \url{https://github.com/franco0209/PWEB2}
		\item URL para el laboratorio 3 en el Repositorio GitHub.	
        \item \url{https://github.com/franco0209/PWEB2/tree/main/lab3}
	\end{itemize}
    \section{EJERCICIO PROPUESTO}
\begin{itemize}
    \item \textcolor{red}{PROBLEMA 01}
    
    Escriba una función que reciba el número de día de la fecha actual \href{https://www.w3schools.com/jsref/jsref_obj_date.asp}{new Date()} y devuelva el texto del día de la semana correspondiente. Por ejemplo, si recibe 0, devolvería “Domingo”.
    
    \item Solución:
    \begin{verbatim}
    <!DOCTYPE html>
    <html>
    <head>
        <title>JavaScript Dates</title>
    </head>
    <body>
        <h1>JavaScript Dates</h1>
        <h2>The new Date() Method</h2>

        <p>Create a date using ISO notation:</p>

        <p id="demo"></p>

        <script>
            function obtenerDiaSemana(numeroDia) {
                const diasSemana = ['Domingo', 'Lunes', 'Martes', 'Miércoles', 'Jueves', 'Viernes', 'Sábado'];
                const fechaActual = new Date();
                const diaActual = fechaActual.getDay(); // Devuelve un número del 0 al 6 representando el día de la semana
                const diaDeseado = (diaActual + numeroDia) % 7;
                return diasSemana[diaDeseado];
            }

            const time = obtenerDiaSemana(0);
            document.getElementById("demo").innerHTML = time;
        </script>

    </body>
    </html>
    \end{verbatim}

\end{itemize}
\begin{itemize}
    \item \textcolor{red}{PROBLEMA 02}
    
    Escriba una página web que reciba un texto y al presionar un botón muestre el mismo texto invertido en otra sección (div). Por ejemplo si se escribe “Hola”, se mostraría como “aloH”.
    
    \item Solución:
    \begin{verbatim}
    <!DOCTYPE html>
    <html>
    <body>
        <h1>JavaScript Text</h1>
        <input type="text" id="texto" placeholder="Escribe una palabra...">
        <button id="boton" onclick="boton()">Haz clic aquí para mostrar la palabra invertida: </button>
        <p id="demo"></p>
        <script>
            function invertirTexto(texto) {
                let invertido = "";
                for (let i = texto.length - 1; i >= 0; i--) {
                    invertido = invertido + texto[i];
                }
                return invertido;
            }
    
            function boton() {
                var txt = document.getElementById("texto").value;
                let texto = invertirTexto(txt);
                document.getElementById("demo").innerHTML = texto;
            }
        </script>
    </body>
    </html>
    
    \end{verbatim}

\end{itemize}
\begin{itemize}
    \item \textcolor{red}{PROBLEMA 03}
    
    Escribir una página que muestre cuántos días faltan para el día de Arequipa!
    
    \item Solución:
    \begin{verbatim}
    <!DOCTYPE html>
<html>
<body>
    <h1>JavaScript Text</h1>
    <input type="text" id="dia" placeholder="Escribe el día actual...">
    <p></p>
    <input type="text" id="mes" placeholder="Escribe el mes actual...">
    <p></p>
    <button id="boton" onclick="boton()">Mostrar cuantos días faltan para el día de Arequipa: </button>
    <p id="demo"></p>
    <script>
        function calcularDistancia(dia, mes) {
            try {
            	var aqp=new Date(2024,7,15);
                var hoy=new Date(2024,mes-1,dia);
                var diff=((aqp-hoy)/(1000*60*60*24));
                if(diff<0){
                	diff=365+diff;
                }
                return diff;
            } catch (error) {
                return "Datos inválidos";
            }
        }

        function boton() {
            var day = parseInt(document.getElementById("dia").value, 10);
            var month = parseInt(document.getElementById("mes").value, 10);

            let texto = "Faltan " +calcularDistancia(day, month) +" días para el aniversario de Arequipa";
            document.getElementById("demo").innerHTML = texto;
        }
    </script>
</body>
</html>

    
    \end{verbatim}

\end{itemize}
\begin{itemize}
    \item \textcolor{red}{PROBLEMA 04}
    
    Escribir una página que muestre cuántos días faltan para el día de Arequipa!
    
    \item Solución:
    \begin{verbatim}
    <!DOCTYPE html>
<html>
<body>
    <h1>JavaScript Text</h1>
    <input type="text" id="dia" placeholder="Escribe el día actual...">
    <p></p>
    <input type="text" id="mes" placeholder="Escribe el mes actual...">
    <p></p>
    <button id="boton" onclick="boton()">Mostrar cuantos días faltan para el día de Arequipa: </button>
    <p id="demo"></p>
    <script>
        function calcularDistancia(dia, mes) {
            try {
            	var aqp=new Date(2024,7,15);
                var hoy=new Date(2024,mes-1,dia);
                var diff=((aqp-hoy)/(1000*60*60*24));
                if(diff<0){
                	diff=365+diff;
                }
                return diff;
            } catch (error) {
                return "Datos inválidos";
            }
        }

        function boton() {
            var day = parseInt(document.getElementById("dia").value, 10);
            var month = parseInt(document.getElementById("mes").value, 10);

            let texto = "Faltan " +calcularDistancia(day, month) +" días para el aniversario de Arequipa";
            document.getElementById("demo").innerHTML = texto;
        }
    </script>
</body>
</html>

    
    \end{verbatim}

\end{itemize}

  \section{W3SCHOOLS}
  \begin{figure}[h]
  \centering
  \includegraphics[width=1.5\textwidth]{Screenshot (23).png}
  \label{fig:imagen_ejemplo}
\end{figure}

  
			
\end{document}