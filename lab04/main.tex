\documentclass{article}
\usepackage[top=3cm, bottom=3cm, outer=3cm, inner=3cm]{geometry}
\usepackage{multicol}
\usepackage{graphicx}
\usepackage{url}
%\usepackage{cite}
\usepackage{hyperref}
\usepackage{array}
\usepackage{natbib}
\usepackage{pdfpages}
\usepackage{multirow}
\usepackage[normalem]{ulem}
\useunder{\uline}{\ul}{}
\usepackage{svg}
\usepackage{xcolor}
\usepackage{listings}
\lstdefinestyle{ascii-tree}{
    literate={├}{|}1 {─}{--}1 {└}{+}1 
}
\lstset{basicstyle=\ttfamily,
    showstringspaces=false,
    commentstyle=\color{red},
    keywordstyle=\color{blue}
}
%\usepackage{booktabs}
\usepackage{caption}
\usepackage{subcaption}
\usepackage{float}
\usepackage{array}

\newcolumntype{M}[1]{>{\centering\arraybackslash}m{#1}}
\newcolumntype{N}{@{}m{0pt}@{}}

%%%%%%%%%%%%%%%%%%%%%%%%%%%%%%%%%%%%%%%%%%%%%%%%%%%%%%%%%%%%%%%%%%%%%%%%%%%%
%%%%%%%%%%%%%%%%%%%%%%%%%%%%%%%%%%%%%%%%%%%%%%%%%%%%%%%%%%%%%%%%%%%%%%%%%%%%
\newcommand{\itemEmail}{fcahua@unsa.edu.pe}
\newcommand{\itemStudent}{Franco Jesus Cahua Soto}
\newcommand{\itemCourse}{Programación Web 2}
\newcommand{\itemCourseCode}{20222171}
\newcommand{\itemSemester}{I}
\newcommand{\itemUniversity}{Universidad Nacional de San Agustín de Arequipa}
\newcommand{\itemFaculty}{Facultad de Ingeniería de Producción y Servicios}
\newcommand{\itemDepartment}{Departamento Académico de Ingeniería de Sistemas e Informática}
\newcommand{\itemSchool}{Escuela Profesional de Ingeniería de Sistemas}
\newcommand{\itemAcademic}{2043 - A}
\newcommand{\itemInput}{}
\newcommand{\itemOutput}{}
\newcommand{\itemPracticeNumber}{5}
\newcommand{\itemTheme}{Python}
%%%%%%%%%%%%%%%%%%%%%%%%%%%%%%%%%%%%%%%%%%%%%%%%%%%%%%%%%%%%%%%%%%%%%%%%%%%%
%%%%%%%%%%%%%%%%%%%%%%%%%%%%%%%%%%%%%%%%%%%%%%%%%%%%%%%%%%%%%%%%%%%%%%%%%%%%

\usepackage[english,spanish]{babel}
\usepackage[utf8]{inputenc}
\AtBeginDocument{\selectlanguage{spanish}}
\renewcommand{\figurename}{Figura}
\renewcommand{\refname}{Referencias}
\renewcommand{\tablename}{Tabla}
\AtBeginDocument{%
    \renewcommand\tablename{Tabla}
}

\usepackage{fancyhdr}
\pagestyle{fancy}
\fancyhf{}
\setlength{\headheight}{30pt}
\renewcommand{\headrulewidth}{1pt}
\renewcommand{\footrulewidth}{1pt}
\fancyhead[L]{\raisebox{-0.2\height}{\includegraphics[width=3cm]{logo_episunsa.png}}}
\fancyhead[C]{\fontsize{7}{7}\selectfont \itemUniversity \\ \itemFaculty \\ \itemDepartment \\ \itemSchool \\ \textbf{\itemCourse}}
\fancyhead[R]{\raisebox{-0.2\height}{\includegraphics[width=1.2cm]{abet.png}}}
\fancyfoot[L]{Estudiante Franco Jesus Cahua Soto}
\fancyfoot[C]{\itemCourse}
\fancyfoot[R]{Página \thepage}

% para el codigo fuente
\usepackage{color, colortbl}
\definecolor{dkgreen}{rgb}{0,0.6,0}
\definecolor{gray}{rgb}{0.5,0.5,0.5}
\definecolor{mauve}{rgb}{0.58,0,0.82}
\definecolor{codebackground}{rgb}{0.95, 0.95, 0.92}
\definecolor{tablebackground}{rgb}{0.8, 0, 0}

\lstset{frame=tb,
    language=bash,
    aboveskip=3mm,
    belowskip=3mm,
    showstringspaces=false,
    columns=flexible,
    basicstyle={\small\ttfamily},
    numbers=none,
    numberstyle=\tiny\color{gray},
    keywordstyle=\color{blue},
    commentstyle=\color{dkgreen},
    stringstyle=\color{mauve},
    breaklines=true,
    breakatwhitespace=true,
    tabsize=3,
    backgroundcolor= \color{codebackground},
}

\begin{document}

\vspace*{10px}

\begin{center}
    \fontsize{17}{17} \textbf{Informe de Laboratorio \itemPracticeNumber}
\end{center}
\centerline{\textbf{\Large Tema: \itemTheme}}

\begin{flushright}
    \begin{tabular}{|M{2.5cm}|N|}
        \hline 
        \rowcolor{tablebackground}
        \color{white} \textbf{Nota}  \\
        \hline 
        \\[30pt]
        \hline             
    \end{tabular}
\end{flushright}   

\begin{table}[H]
    \centering
    \begin{tabular}{|M{4.7cm}|M{4.8cm}|M{4.8cm}|}
        \hline 
        \rowcolor{tablebackground}
        \color{white} \textbf{Estudiante} & \color{white}\textbf{Escuela}  & \color{white}\textbf{Asignatura}   \\
        \hline 
        \itemStudent & \itemSchool & \itemCourse \\
        \itemEmail & & Semestre: \itemSemester \\
        & & Código: \itemCourseCode \\
        \hline             
    \end{tabular}
\end{table}     

\begin{table}[H]
    \centering
    \begin{tabular}{|M{4.7cm}|M{4.8cm}|M{4.8cm}|}
        \hline 
        \rowcolor{tablebackground}
        \color{white}\textbf{Laboratorio} & \color{white}\textbf{Tema}  & \color{white}\textbf{Duración}   \\
        \hline 
        \itemPracticeNumber & \itemTheme & 04 horas   \\
        \hline 
    \end{tabular}
\end{table}

\begin{table}[H]
    \centering
    \begin{tabular}{|M{4.7cm}|M{4.8cm}|M{4.8cm}|}
        \hline 
        \rowcolor{tablebackground}
        \color{white}\textbf{Semestre académico} & \color{white}\textbf{Fecha de inicio}  & \color{white}\textbf{Fecha de entrega}   \\
        \hline 
        \itemAcademic & \itemInput &  \itemOutput  \\
        \hline 
    \end{tabular}
\end{table}

\section{TAREA}

\section{URL DE REPOSITORIO GITHUB}
\begin{itemize}
    \item URL para el Repositorio GitHub.
    \item \url{https://github.com/franco0209/PWEB2}
    \item URL para el laboratorio 3 en el Repositorio GitHub.
    \item \url{https://github.com/franco0209/PWEB2/tree/main/lab04}
\end{itemize}

\section{EJERCICIO PROPUESTO}
\begin{itemize}
    \item \textcolor{red}{PROBLEMA 01}
    \item Imagen 1
    \item Solución:
    \begin{lstlisting}
from chessPictures import *
from interpreter import draw
blackKn=knight.negative()
figureUp =knight.join(blackKn)
figureDown=blackKn.join(knight)
ima1=figureDown.up(figureUp)
draw(ima1)
    \end{lstlisting}
    \includegraphics[width=0.5\textwidth]{i1.png}
\end{itemize}

\begin{itemize}
    \item \textcolor{red}{PROBLEMA 02}
    \item Imagen 2
    \item Solución:
    \begin{lstlisting}
from chessPictures import *
from interpreter import draw
blackKn=knight.negative()
blackKnRY=blackKn.verticalMirror()
figureUp =knight.join(blackKn)
figureDown=blackKnRY.join(knight)
ima1=figureDown.up(figureUp)
draw(ima1)
    \end{lstlisting}
    \includegraphics[width=0.5\textwidth]{i2.png}
\end{itemize}

\begin{itemize}
    \item \textcolor{red}{PROBLEMA 03}
    \item Imagen 3
    \item Solución:
    \begin{lstlisting}
from chessPictures import *
from interpreter import draw
ima3=queen.horizontalRepeat(3)
draw(ima3)
    \end{lstlisting}
    \includegraphics[width=0.5\textwidth]{i3.png}
\end{itemize}

\begin{itemize}
    \item \textcolor{red}{PROBLEMA 04}
    \item Imagen 4
    \item Solución:
    \begin{lstlisting}
from chessPictures import *
from interpreter import draw
squareN=square.negative()
figureDoble=square.join(squareN)
ima4=figureDoble.horizontalRepeat(3)
draw(ima4)
    \end{lstlisting}
    \includegraphics[width=0.5\textwidth]{i4.png}
\end{itemize}

\begin{itemize}
    \item \textcolor{red}{PROBLEMA 05}
    \item Imagen 5
    \item Solución:
    \begin{lstlisting}
from chessPictures import *
from interpreter import draw
squareN=square.negative()
figureDoble=square.join(squareN)
ima4=figureDoble.horizontalRepeat(3)
ima5=ima4.verticalMirror()
draw(ima5)
    \end{lstlisting}
    \includegraphics[width=0.5\textwidth]{i5.png}
\end{itemize}

\begin{itemize}
    \item \textcolor{red}{PROBLEMA 06}
    \item Imagen 6
    \item Solución:
    \begin{lstlisting}
from chessPictures import *
from interpreter import draw
squareN=square.negative()
figureDoble=square.join(squareN)
ima4=figureDoble.horizontalRepeat(3)
ima5=ima4.verticalMirror()
ima45=ima5.up(ima4)
ima6=ima45.up(ima45)
draw(ima6)
    \end{lstlisting}
    \includegraphics[width=0.5\textwidth]{i6.png}
\end{itemize}

\begin{itemize}
    \item \textcolor{red}{PROBLEMA 07}
    \item Imagen 7
    \item Solución:
    \begin{lstlisting}
from chessPictures import *
from interpreter import draw
squareN=square.negative()
squares2=square.join(squareN)
row=squares2.horizontalRepeat(3)
rowN=row.negative()
rows4=(rowN.up(row)).verticalRepeat(1)

midPieces1=(rock.join(knight)).join(bishop)
midPieces2=(bishop.join(knight)).join(rock)
midPieces=midPieces1.join(queen).join(king).join(midPieces2)
pawns=pawn.horizontalRepeat(7)
pawnsT=row.under(pawns)
midPiecesT=rowN.under(midPieces)

# BLANCAS
whites=midPiecesT.up(pawnsT)
# NEGRAS
blacks=(pawnsT.up(midPiecesT)).negative()

final=whites.up(rows4).up(blacks)
draw(final)
    \end{lstlisting}
    \includegraphics[width=0.5\textwidth]{i7.png}
\end{itemize}

\end{document}
